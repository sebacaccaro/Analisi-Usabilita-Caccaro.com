\textit{A che pagina sono arrivato?}\newline
Caccaro produce \textit{mobili}, ma questa parola non viene menzionata nella homepage nemmeno una volta. Ricorre invece in quasi ogni paragrafo la parola \textit{spazi} (17 volte), che sostanzialmente è il tema dettato dal motto dell'azienda. Un visitatore che non conoscesse l'azienda a primo acchito potrebbe essere portato a pensare si tratti di un generico sito di arredamento (ad esempio, il sito di un designer di interni), guidato anche dalle generiche immagini di ambienti arredati presenti nel \textit{first cut}.\\
Con un po' di scrolling è possibile avere un'idea più chiara sui prodotti e capire che si tratta effettivamente di un'azienda che vende mobili. 
\\Detto ciò, l'asse \textit{where} poteva essere sicuramente espresso in modo più efficace: basti pensare che il vecchio slogan di Caccaro, "\textit{Mobili per passione}", messo sotto il logo avrebbe fornito immediatamente le informazioni per questo asse informativo. 
