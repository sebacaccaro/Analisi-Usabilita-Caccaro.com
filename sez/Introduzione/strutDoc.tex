L'analisi del sito è strutturata in 4 sezioni principali:
\begin{itemize}
	\item \textbf{Analisi delle pagine:} Verranno analizzate alcune pagine significative del sito. Per ogni pagina verranno coperti i seguenti punti:
	\begin{itemize}
	 	\item \textit{Assi informativi:} Verranno valutate la presenza e la qualità degli assi informativi.
	 	\item \textit{Considerazioni generali:} Verranno presi in analisi il layout e le scelte di design della pagina.
	 	\item \textit{Mobile:} Verrà brevemente analizzata la versione mobile della pagina, evidenziandone le differenze più significative con la versione desktop.
	 	\item \textit{In breve:} Verranno riassunti i principali punti positivi o negativi della pagina.
	 \end{itemize} 
	 \item \textbf{Elementi comuni:} Saranno analizzati in modo approfondito elementi presenti in ogni pagina, come la navbar, onde evitare di ripetersi in ogni sezione.
	 \item \textbf{Varie:} Considerazioni sul sito che non sono strettamente riconducibili all'analisi di una pagina.
	 \item \textbf{Considerazioni finali:} Verranno tirate le somme dell'analisi del sito. Verrà, inoltre, assegnato un voto in decimi.
\end{itemize}